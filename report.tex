\documentclass[12pt]{article}
\usepackage{amsmath}
\usepackage{amssymb}
\usepackage{amsthm}
\usepackage{fancyhdr}
\usepackage{graphicx}
\pagestyle{fancy}
\lhead{SID: 309201470}
\rhead{SID: 310------}

\title{COMP3608: Othello}
\author{Elizaveta Lisa Fedorenko: 309201470
\\ Tianyu Pu: 310------}
\date{\today}
\fancyhead[L]{309201470}
\fancyhead[R]{310------}

\begin{document}
\maketitle

\section{Implementation of Strategies}
\subsection{Heuristic}
The game of Othello is complex [INSERT STATISTIC AND REFERENCE]. We felt that each strategy didn't differ sufficiently to warrant a seperate evalutation function. Rather we felt that having a more informed evaluation function for each strategy was optimal. There are three main methods to judge a position in Othello as follows [INSERT REFERENCE]:
\begin{enumerate}
\item \emph{Number of tokens} Obviously as the aim of the game is to have the most tokens, this is a good basic estimate of who is winning
\item \emph{Stability} Certain pieces are a lot easier to 'flip' than others. In particular corners are very stable and cannot be flipped. Pieces protruding from corners also are more stable.
\item \emph{mobility} Having more moves available means you are better positioned strategically and less constricted.
\end{enumerate}
Each of these methods has its own shortcomings and advantages [INSERT REFERENCE]. We therefore chose to implement a heuristic that weighted each method appropriately. Thus creating a much 'smarter' AI.
Our heuristic can be represented thus:
\begin{equation}
h(\mbox{black}) = \mbox{number black pieces}*f(\mbox{pieces}) + \mbox{factor}*\mbox{number available moves}
\end{equation}
Where f() is a function of the stability of each piece. f() weights corner pieces and pieces joint to corner pieces higher than centre unprotected pieces. The factor is a constant which chooses the balance to be given between each of the strategy calculations.
\subsection{Strategy A}
\subsection{Strategy B}
\subsection{Strategy C}
\section{Findings}
\subsection{Results}
\subsection{Conclusion}
\section{Reflection}
\section{Appendix}
\subsection{Strategy A}
\subsection{Strategy B}
\subsection{Strategy C}
\end{document}
